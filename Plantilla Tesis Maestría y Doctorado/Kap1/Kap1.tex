\chapter{Introducción}
\begin{center}
	En este capítulo se presenta una introducción a las formas\\ 		
	cuadráticas, sus representaciones y su clasificación.
\end{center}

\section{FORMAS CUADRÁTICAS}

\paragraph*{}
Esta sección fue adaptada de \citep{LayDavidC2001Alys}. Algunas demostraciones aquí omitidas pueden consultarse en dicha referencia.\\
Fijamos un conjunto de variables $\{x_1, x_2, \ldots, x_n\}$. Un \textbf{monomio} es un producto de la forma $x_{1}^{e_{1}} \cdot x_{2}^{e_{2}} \cdots x_{n}^{e_{n}}$ donde cada $e_{i}$ es un número natural. Un \textbf{término} se forma al multiplicar a un monomio por una constante, a la cual llamamos \textbf{coeficiente} del término. Un \textbf{polinomio} es una suma finita de términos.\\
Ejemplo: $7x^{2}y + 5x^z{4} - y$ es un polinomio sobre las variables $x$, $y$ y $z$ con tres monomios: $x^{2}y$, $xz^{4}$ y $y$ cuyos respectivos coeficientes son $2$, $4$, $1$ respectivamente.\\
Una \textbf{forma cuadrática} es un polinomio en el que $q : \mathbb{R}^{n} \rightarrow \mathbb{R}$(con $n > 0$) en el que cada monomio del mismo es una variable al cuadrado o la multiplicación de dos variables. Esto es equivalente a decir que $q$ se puede expresar como

\begin{equation*}
q(x_{1}, x_{2}, \ldots, x_{n}) = \sum_{i=1}^{n} q_{ii}x_{i}^{2}+ \sum_{j=2}^{n}\sum_{i=1}^{j-1} q_{ij}x_{i}x_{j}
\end{equation*}

\paragraph*{}
Los ejemplos más usuales aparecen al lado izquierdo del signo igual de las ecuaciones de las cónicas con centro en el origen

\begin{equation*}
ax^{2} + 2bxy + cy^{2} = d
\end{equation*}

\paragraph*{}
y de las superficies cuadráticas con centro en el origen

\begin{equation*}
ax^{2} + 2dxy + 2exz + by^{2} + 2fyz + cz^{2} = g
\end{equation*}

\paragraph*{}
donde $a$, $b$, $c$, $d$, $e$, $f$ y $g$ son números reales. Este tipo de polinomios surgen de manera natural en diversas áreas de la ingeniería, procesamiento de señales, cinética, economía, geometría diferencial y estadística.\\
En algunos curso básicos de álgebra lineal se suele definir el concepto de forma cuadrática como una función que se puede escribir como

\begin{equation*}
q(\overrightarrow{x}) = \frac{1}{2} \overrightarrow{x}^{T} A \overrightarrow{x}
\end{equation*}

\paragraph*{}
para alguna matriz simétrica $A$. En realidad esta representación matricial es equivalente a nuestra definición con monomios; a continuación veremos como pasar de una representación a otra. Primero notemos que $\frac{1}{2} \overrightarrow{x}^{T} A \overrightarrow{x}$ se puede reescribir como sigue:

\begin{align*}
\frac{1}{2} \overrightarrow{x}^{T} A \overrightarrow{x} &= \frac{1}{2}[x_{1}, x_{2},\ldots,x_{n}] 
\begin{bmatrix}
a_{11} & a_{12} & \cdots & a_{1n}\\
a_{21} & a_{22} & \cdots & a_{2n}\\
\vdots & \vdots & \ddots & \vdots \\
a_{n1} & a_{n2} & \cdots & a_{nn}
\end{bmatrix} 
\begin{bmatrix}
x_{1}\\
x_{2}\\
\vdots\\
x_{n}
\end{bmatrix} \\ 
&= \frac{1}{2} \left[\sum_{i=1}^{n} a_{i1}x_{i}, \sum_{i=1}^{n} a_{i2}x_{i}, \ldots , \sum_{i=1}^{n} a_{in}x_{i} \right] 
\begin{bmatrix}
x_{1}\\
x_{2}\\
\vdots\\
x_{n}
\end{bmatrix} \\ 
&=   \frac{1}{2} \left[\sum_{i=1}^{n} a_{i1}x_{i}x_{1} + \cdots + \sum_{i=1}^{n} a_{in}x_{i}x_{n}\right]\\ 
&=   \frac{1}{2} \sum_{i=1}^{n}\sum_{j=1}^{n} a_{ij}x_{i}x_{j}
\end{align*}

\paragraph*{}
Al desarrollar esto se puede ver que el coeficiente de $x_{i}$ $x_{j}$ es $q_{ij} = \frac{1}{2} \left(a_{ij} + a_{ji} \right)$ porque $x_{i}$ $x_{j}$ = $x_{j}$ $x_{i}$, pero como dijimos que $A$ es simétrica $\left( a_{ij} = a_{ji}\right)$ entonces $q_{ij} = a_{ij}$. En particular cuando $i = j$ se tiene que el coeficiente de $x_{i}^{2}$ es $q_{ii} = \frac{1}{2}a_{ii}$. Por lo cuál tenemos la siguiente identidad:

\begin{equation}
    \frac{1}{2} \overrightarrow{x}^{T} A \overrightarrow{x} = \sum_{i=1}^{n}\frac{1}{2} a_{ii}x_{i}^{2} + \sum_{j=2}^{n}\sum_{i=1}^{j-1} a_{ij}x_{i}x_{j}
    \label{ecuacion:1.1}
\end{equation}

\paragraph*{}
La matriz simétrica $A$ de esta identidad se conoce como \textbf{matriz asociada a la forma cuadrática $q$}. Se denota por \textbf{$A_{q}$}, y según la ecuación  \ref{ecuacion:1.1} se puede calcular como sigue:

\begin{equation}
a_{ij} = \left \{ 
    \begin{matrix} 
    q_{ij} & \mbox{si } i < j\\
    2q_{ij} & \mbox{si } i = j\\ 
    q_{ji} & \mbox{si } i > j
    \end{matrix}\right.
    \label{ecuacion:1.2}
\end{equation}

\paragraph*{}
Por ejemplo, para $q(x, y, x) = 6x^{2} + 2y^{2} + 4z^{2} - 2xy + 10xz - 6yz$

\begin{equation*}
q(x. y. z) = \frac{1}{2}\left[x ~  y ~ z\right] 
\begin{bmatrix}
12 & -2 & 10\\
-2 & 4 & -6\\
10 & -6 & 8
\end{bmatrix}
\begin{bmatrix}
x\\
y\\
z\\
\end{bmatrix}
\end{equation*}

\paragraph*{}
\textbf{CAMBIO DE VARIABLE}

\paragraph*{}
Un \textbf{cambio de variable} es una ecuación de la forma

\begin{equation}
    \begin{matrix} 
    \overrightarrow{x} = P\overrightarrow{y} & \mbox{o bien } & \overrightarrow{y} = P^{-1}\overrightarrow{x}\\
    \end{matrix}.
    \label{ecuacion:1.3}
\end{equation}

\paragraph*{}
donde $P$ es una matriz invertible y $\overrightarrow{y}$ es un nuevo vector variable en $\mathrm{R}^{n}$. Si se aplica el cambio de variable \ref{ecuacion:1.3} sobre una forma cuadrática $q(\overrightarrow{x}) = \frac{1}{2}\overrightarrow{x}^{T}A\overrightarrow{x}$ se obtiene una matriz cuadrática $q'(\overrightarrow{y})$ cuya matriz asociada es $P^{T} ~ A ~ P$:

\begin{equation*}
    \frac{1}{2}\overrightarrow{x}^{T}\overrightarrow{x} = \frac{1}{2}\left(P\overrightarrow{y}\right)^{T}~A~\left(P\overrightarrow{y}\right) = \frac{1}{2}\left(\overrightarrow{y}^{T} P^{T}\right)~A~\left(P\overrightarrow{y}\right) = \frac{1}{2}\overrightarrow{y}^{T}\left(P^{T}~A~P\right)\overrightarrow{y}
\end{equation*}.

\paragraph*{}
Garantizamos que $P^{T}~A~P$ es una matriz simétrica (y por tanto que en verdad corresponde a otra forma cuadrática) porque

\begin{eqnarray*}
\left(P^{T}~A~P\right)^{T}&=&P^{T}~A^{T}~\left(P^{T}\right)^{T}\\
&=&P^{T}~A^{T}~P\\
&=&P^{T}~A~P
\end{eqnarray*}

\paragraph*{}
Mediante el cambio de variable $\overrightarrow{y} = P~\overrightarrow{x}$ tenemos que $q(\overrightarrow{x}) = q'(\overrightarrow{y})$ y diremos que $q$ y $q'$ son \textbf{equivalentes} mediante la matriz invertible $P$. Si denotamos con $L(\overrightarrow{x})$ a la transformación dada por, $L(\overrightarrow{x}) = P\overrightarrow{x}$ y tomamos $\overrightarrow{y} = L(\overrightarrow{x})$, entonces $q'(\overrightarrow{y}) = q'\left(L(\overrightarrow{x})\right) = \left(q' \circ L\right)(\overrightarrow{x})$, por lo tanto

\begin{equation*}
q' = q \circ L
\end{equation*}

\begin{example}
Consideremos la forma cuadrática $x^{2} - 5y^{2} - 8xy$ con matriz asociada
\begin{equation*}
    \begin{matrix} 
    A = \begin{bmatrix}
2 & -8\\
-8 & 10
\end{bmatrix}
    \end{matrix}.
\end{equation*}
y el cambio de variable definido por
\begin{equation*}
    \begin{matrix} 
    \begin{bmatrix}
        x\\
        y
    \end{bmatrix} = \begin{bmatrix}
    \frac{2}{\sqrt(5)} & \frac{1}{\sqrt(5)}\\
    -\frac{1}{\sqrt(5)} & \frac{2}{\sqrt(5)}
    \end{bmatrix}\begin{bmatrix}
    z\\
    w
    \end{bmatrix} = \begin{bmatrix}
        \frac{2z + w}{\sqrt(5)}\\
        \frac{-z + 2w}{\sqrt(5)}
        \end{bmatrix}
    \end{matrix}.
\end{equation*}

La matriz $ \begin{matrix} 
    P = \begin{bmatrix}
\frac{2}{\sqrt(5)} & \frac{1}{\sqrt(5)}\\
-\frac{1}{\sqrt(5)} & \frac{2}{\sqrt(5)}
\end{bmatrix}
    \end{matrix}$ es invertible, y su inversa es 

\begin{equation*}
    \begin{matrix} 
    P^{-1} = \begin{bmatrix}
\frac{2}{\sqrt(5)} & -\frac{1}{\sqrt(5)}\\
\frac{1}{\sqrt(5)} & \frac{2}{\sqrt(5)}
\end{bmatrix}
    \end{matrix} = P^{T}
\end{equation*}

Calculamos $ \begin{matrix} 
    P^{T}~A~P = \begin{bmatrix}
6 & 0\\
0 & -14
\end{bmatrix}
    \end{matrix}$ para concluir que 
\begin{equation*}
x^{2} - 5y^{2} - 8xy = 3z^{2} - 7w^{2}
\end{equation*}

Esto es lo que se hubiese obtenido de hacer las sustituciones 

\begin{eqnarray*}
x & \leftarrow & \frac{2z + w}{\sqrt(5)}\\
y & \leftarrow & \frac{-z + 2w}{\sqrt(5)}
\end{eqnarray*}

sobre la expresión $x^{2} - 5y^{2} - 8xy$.
\end{example}

\paragraph*{}
Las cosas a destacar en este ejemplo son:
\begin{itemize}
    \item $P$ resultó ser una \textbf{matriz ortogonal}, en otras palabras, $P^{-1} = P^{T}$.
    \item $A$ es una matriz \textbf{ortogonalmente diagonizable}, en otras palabras, existe una matriz ortogonal $P$ tal que $P^{-1}~A~P$ es una matriz diagonal.
\end{itemize}

\begin{theorem}
Toda matriz es ortogonalmente diagonizable si y solo si es simétrica
\label{teorema:1.1}
\end{theorem}

\paragraph*{}
El teorema \ref{teorema:1.1} se dejara sin demostración. Lo importante es que, a partir de este teorema, y dado que las matrices asociadas a las formas cuadráticas son simétricas, se sigue el siguiente resultado:

\begin{theorem}(de los ejes principales).
Toda forma cuadrática $q(\overrightarrow{x})$ es equivalente mediante una matriz ortogonal $P$ a una forma cuadrática 
\begin{equation}
    q'(\overrightarrow{y}) = \lambda_{1}y_{1}^{2} + \lambda_{2}y_{2}^{2} + \cdots + \lambda_{n}y_{n}^{2}
\end{equation}
\label{teorema:1.2}
\end{theorem}

\begin{proof}
La matriz $A$ asociada a $q$ es simétrica, luego por teorema anterior existe una matriz $P$ tal que 
\begin{equation*}
    P^{-1}~A~P = \begin{bmatrix}
    \lambda_{1} & &\\
    & \ddots & \\
    & & \lambda_{n}
    \end{bmatrix}
    \label{ecuacion:1.4}
\end{equation*}

Haciendo $\overrightarrow{x} = P\overrightarrow{y}$ y se obtiene $q'(\overrightarrow{y}) = \frac{1}{2}\overrightarrow{y}^{T}\left(P^{T}~A~P\right)\overrightarrow{y} = \lambda_{1}y_{1}^{2} + \lambda_{2}y_{2}^{2} + \cdots + \lambda_{n}y_{n}^{2}$

\end{proof}

\paragraph*{}
Como se vera, esta representación es muy útil para clasificar formas cuadráticas. Diremos que una forma cuadrática $q$ es \textbf{definida positiva} si $q(\overrightarrow{x}) > 0, ~\forall~ \overrightarrow{x} \neq \overrightarrow{0}$ 

\begin{lemma}
Si dos formas cuadráticas $q(\overrightarrow{x})$ y $q'(\overrightarrow{y})$ son equivalentes mediante el cambio de variable $\overrightarrow{y} = P\overrightarrow{x}$ y si una de ellas es definida positiva entonces la otra también lo es.
\label{lema:1.3}
\end{lemma}

\begin{proof}
Supongamos que $q(\overrightarrow{x})$ es definida positiva y definimos la transformación lineal $L(\overrightarrow{x}) = P\overrightarrow{x}$, entonces $L(\overrightarrow{0}) = L(\overrightarrow{0}+ \overrightarrow{0}) = L(\overrightarrow{0}) + L(\overrightarrow{0})$, de donde $L(\overrightarrow{0}) = \overrightarrow{0}$. Como $P$ es una matriz invertible entonces $L^{-1}(\overrightarrow{x}) = P^{-1}x$. En particular $L$ debe ser inyectiva y por lo tanto $L(\overrightarrow{x})=\overrightarrow{0}$ implica que $\overrightarrow{x}=\overrightarrow{0}$. Si $q$ es definida positiva entonces $q(\overrightarrow{x}) = q'(\overrightarrow{y}) > 0$ para todo $\overrightarrow{y} \neq \overrightarrow{0}$ y se cumple que $\overrightarrow{x} \neq \overrightarrow{0} \Leftrightarrow \overrightarrow{y} = P^{-1}\overrightarrow{x} \neq \overrightarrow{0}$; por lo tanto $q'$ es definida positiva , entonces aplicando el mismo razonamiento con $L(\overrightarrow{y}) = P^{-1}\overrightarrow{y}$ se llega a la conclusión de que $q$ también es definida positiva. 
\end{proof}

\begin{theorem}
Sea $q$ una forma cuadrática y supongamos que 
\begin{equation*}
q \left(\overrightarrow{x}\right) = q'(y) = \lambda_{1}y_{1}^{2} + \lambda_{2}y_{2}^{2} + \cdots + \lambda_{n}y_{n}^{2}
\end{equation*}
entonces $q$ es definida positiva si y solo si todos los $\lambda_{i} > 0$
\label{teorema:1.4}
\end{theorem}

\begin{proof}
Por el lema anterior $q$ es definida positiva si y solo si $q'$ es definida positiva. Por contradicción, si $\lambda_{i} \leq 0$ entonces definimos $y' = (y_{1}, y_{2}, \ldots, y_{n})$ donde todos los $y_{k}=0$ excepto $y_{i} = 1$. Claramente $\overrightarrow{y} \neq \overrightarrow{0}$ pero $q(\overrightarrow{y}) = \lambda_{i} \leq 0$; por lo tanto $q$ no es definida positiva.
\end{proof}

\paragraph{}
Como las formas cuadráticas se pueden representar mediante matrices entonces podemos decir que una matriz $A$ es definida positiva si su forma cuadrática asociada $q(\overrightarrow{x}) = \frac{1}{2}\overrightarrow{x}^{T}A\overrightarrow{x}$ es definida positiva.

\paragraph{}
El siguiente teorema típico de los cursos de análisis numérico nos da un criterio computacionalmente eficiente para decidir cuando una matriz simétrica es definida positiva(sin demostración).

\begin{theorem}
Una matriz $A$ es definida positiva si y solo si tiene \textbf{factorización de Cholesky}, es decir, se puede escribir como $A=R^{T}R$ donde $R$ es una matriz triangula superior con entradas positivas.
\label{teorema:1.5}
\end{theorem}

\paragraph{}
El siguiente teorema es el criterio de Sylvester para que una forma cuadrática sea positiva definida.

\begin{theorem}
Sean $V$ un espacio vectorial real de dimensión $n, q \in \mathcal{Q}(V)$, $B$ una base de $V$ . Denotamos por $q$ a la matriz asociada a la forma cuadrática q respecto a la base $\mathcal{B}$. Entonces las siguientes condiciones son equivalentes:
\begin{enumerate}
\item $q > 0$, esto es, $q(x) > 0$ para todo $x \in V \setminus \{0\}$.
\item todos los menores principales de $qB$ son positivos:
\begin{equation*}
\forall I \subseteq {1, . . . , n} ~~ \delta_{I}(q_{\mathcal{B}}) > 0.
\end{equation*}
\item todos los menores de la matriz $q_{\mathcal{B}}$ son positivos: para todo $k \in {1, \ldots, n}$
\begin{equation*}
\Delta_{k}(q_{B}) > 0.
\end{equation*}
\end{enumerate}
\label{teorema:1.6}
\end{theorem}

\section{FORMAS UNITARIAS}
\paragraph*{}
Esta sección fue adaptada de \citep{Ringel1985TameAA} y \citep{alma991031505829703276}. Las \textbf{formas cuadráticas enteras} son un caso especial de las formas cuadráticas donde todos los coeficientes $q_{ij}$ son todos números enteros. Si además exigimos que $q_{ii} = 1$ para $i = 1, 2, \ldots, n$ entonces las llamamos \textbf{formas cuadráticas unitarias}. En el resto de este trabajo solo se trabaja con formas cuadráticas enteras que son unitarias por lo que  las llamaremos \textbf{formas unitarias}.

\paragraph*{}
\textbf{CAMBIO DE VARIABLE ENTERO}

\paragraph*{}
Una \textbf{matriz entera $M$} es una matriz tal que todas sus entradas son números enteros, y es $\mathbb{Z}$\textbf{-invertible} si además su matriz inversa $M^{-1}$ también es una matriz entera. Un \textbf{cambio de variable entero} es un cambio de variable $\overrightarrow{y} = P\overrightarrow{x}$ donde $P$ es una matriz $\mathbb{Z}$-invertible. Los cambios de variables enteros tienen la propiedad de que transforman formas cuadráticas enteras en formas cuadráticas enteras(demostración: sean $A$ y $P$ matrices enteras, entonces $P^{T}~A~P$ es una matriz entera). Cuando $q(\overrightarrow{x}) = q'(\overrightarrow{y})$ mediante el cambio de variable entero $\overrightarrow{y} = P\overrightarrow{x}$ diremos que $q$ y $q'$ son $\mathbb{Z}$-\textbf{equivalentes} mediante la matriz $\mathbb{Z}$-invertible $P$.

\paragraph*{}
\textbf{Bi-gráficas asociadas a formas unitarias}

\paragraph*{}
Si $q$ es una forma unitaria entonces le asociaremos una bigráfica \textbf{$B$}$_{q}$ construida de la siguiente manera:

\begin{itemize}
    \item Existe un vértice $x_{i}$ para cada variable $x_{i}$
    \item Si $q_{ij} > 0$ entonces trazamos aristas punteadas entre los vértices $x_{i}$ y $x_{j}$ con peso $q_{ij}$.$$x_{i}\overset{q_{ij}}{\cdots\cdots}x_{j}$$
    
    \item Si $q_{ij} < 0$ entonces trazamos aristas sólidas entre los vértices $x_{i}$ y $x_{j}$ con peso $|q_{ij}|$.$$x_{i}\overset{|q_{ij}|}{\rule[1mm]{7mm}{0.1mm}}x_{j}$$
\end{itemize}

\begin{figure}[H]
\begin{center}
 \begin{tikzpicture}
  \node (v0) at (1, 0.) {x};
  \node (v1) at (-1, 0) {y};
  \node (v2) at (0, -1) {z};
  \node (v3) at (0, 1) {w};
  \draw (v0) -- (v1);
  \draw (v1) -- (v2);
  \draw[dotted] (v3) -- (v0);
  \draw (v0) -- (v2);
  \draw (v1) -- (v0);
  \draw[dotted](v1) -- (v3);
  \draw (v2) -- (v0);
  \draw (v2) -- (v3);
  \draw (v3) -- (v2);
\end{tikzpicture}
\caption{Bigráfica asociada a  $wx -yz - xz - xy - wz + wy$}
\label{figura:1.1}
\end{center}
\end{figure}

\paragraph*{}
Este proceso se puede revertir y asociar a toda bigráfica $G$ (posiblemente con aristas punteadas) una forma unitaria que denotaremos por \textbf{$q_{G}$}. Ahora, toda la información de $q_{G}$ está codificada en $G$: La existencia de un vértice $x$ nos dice que la forma cuadrática está definida sobre alguna variable $x$ y que contiene el término $x^{2}$ (porque $q$ es una forma cuadrática unitaria). El coeficiente del monomio $xy$ es $c = a_{p} - a_{s}$ donde $a_{p}$ es la cantidad de aristas punteadas entre $x$ y $y$, y $a_{s}$ es la cantidad de aristas sólidas entre estos mismos vértices. Más aún cabe recalcar que nosotros estamos descartando gráficas con lazos, por lo que cualquier gráfica en verdad define a una forma cuadrática unitaria.\\
Las gráficas de Dynkin se presentan en la figura \ref{figura:1.2}. Hay que resaltar que las gráficas $\DynD_{n}$ están definidas para $n \geq 4$ mientras que las gráficas $\DynE_{n}$ solo se definen para $n = 6, 7, 8$. Lo importante de estas gráficas radica en que permite dar una caracterización elegante de las formas unitarias que son definidas positivas tal como se explica a continuación.

\begin{figure}[H]
\begin{center}
    \begin{tabular}{ll}
    \hline 
     Notación \vline & Diagrama de Dynkin \\ 
    \hline
    $\DynA_{n}$, $n\ge1$ &
    \begin{tikzpicture}
    [baseline=(v1.base)]
    \node (v1) at (0, 0) {$1$};
    \node (v2) at (1, 0) {$2$};
    \node (v3) at (4, 0) {$n$};
    \node[draw = none] (v5) at (2.5, 0) {$\ldots$};
    \draw (v1) -- (v2) -- (2, 0);
    \draw (3, 0) -- (v3);
    \end{tikzpicture}\\
    \newline
    $\DynD_{n}$, $n\ge4$ &
    \begin{tikzpicture}
    [baseline=(v1.base)]
    \node (v1) at (0, 0) {$2$};
    \node (v2) at (1, 0) {$3$};
    \node (v3) at (2, 0) {$4$};
    \node (v4) at (5, 0) {$n$};
    \node (v5) at (1, 1) {$1$};
    \node[draw = none] (dots) at (3.5, 0) {$\ldots$};
    \draw (v1) -- (v2) -- (v3) -- (3, 0); \draw (v5)
    -- (v2); \draw (4, 0) -- (v4);
    \end{tikzpicture} \\
    \newline
    $\DynE_{n}$, $n=6$ &
    \begin{tikzpicture} [baseline=(v1.base)]
    \node (v1) at (0, 0) {$2$};
    \node (v2) at (1, 0) {$3$};
    \node (v3) at (2, 0) {$4$};
    \node (v4) at (3, 0) {$5$};
    \node (v5) at (4, 0) {$6$};
    \node (v6) at (2, 1) {$1$};
    \draw (v1) -- (v2) -- (v3) -- (v4);
    \draw (v6) -- (v3);
    \draw (v5) -- (v4);
    \end{tikzpicture}\\
    \newline
    $\DynE_{n}$, $n=7$ &
    \begin{tikzpicture} [baseline=(v1.base)]
    \node (v1) at (0, 0) {$2$};
    \node (v2) at (1, 0) {$3$};
    \node (v3) at (2, 0) {$4$};
    \node (v4) at (3, 0) {$5$};
    \node (v5) at (4, 0) {$6$};
    \node (v7) at (5, 0) {$7$};
    \node (v6) at (2, 1) {$1$};
    \draw (v1) -- (v2) -- (v3) -- (v4);
    \draw (v6) -- (v3);
    \draw (v4) -- (v5);
    \draw (v5) -- (v7);
    \draw (v6) -- (v3);
    \end{tikzpicture}\\
    \newline
    $\DynE_{n}$, $n=8$ &
    \begin{tikzpicture} [baseline=(v1.base)]
    \node (v1) at (0, 0) {$2$};
    \node (v2) at (1, 0) {$3$};
    \node (v3) at (2, 0) {$4$};
    \node (v4) at (3, 0) {$5$};
    \node (v5) at (4, 0) {$6$};
    \node (v7) at (5, 0) {$7$};
    \node (v8) at (6, 0) {$8$};
    \node (v6) at (2, 1) {$1$};
    \draw (v1) -- (v2) -- (v3) -- (v4);
    \draw (v6) -- (v3);
    \draw (v4) -- (v5);
    \draw (v5) -- (v7);
    \draw (v7) -- (v8);
    \end{tikzpicture}
    \end{tabular} 
    \caption{Gráficas de Dynkin. El subíndice $n$ indica la cantidad de vértices que tiene la bigráfica.}
    \label{figura:1.2}
\end{center}
\end{figure}

\paragraph*{}
Toda forma unitaria $q$ manejada en este trabajo es definida positiva.

\begin{theorem}
Toda forma unitaria $B_{q}$ es conexa, así $q$ es $\mathbb{Z}$-equivalente a una forma unitaria cuya bigráfica asociada es un diagrama de Dynkin.
\label{teorema:1.6}
\end{theorem}

\paragraph{}
Para poder hacer la demostración hay que comprender el teorema. El teorema dice que la forma unitaria y conexa $q(\overrightarrow{x}) = \frac{1}{2}\overrightarrow{x}^{T}A\overrightarrow{x}$ es definida positiva si, y sólo si, se puede llevar, mediante un cambio de variable entero $\overrightarrow{y} = P\overrightarrow{x}$, a la forma $q'(\overrightarrow{y}) = \frac{1}{2}\overrightarrow{y}^{T}\left(P^{T}~A~P\right)\overrightarrow{y}$ donde $\textbf{B}_{q'}$, tiene la propiedad de que cada una de sus componentes conexas es una gráfica de Dynkin. A dicha gráfica $\textbf{B}_{q'}$ se le llama el \textbf{tipo Dynkin} de $q$.

\begin{example}\citep{AbarcaSoteloMarioAlberto2011Apds}
La forma unitaria
\begin{equation}
    q(w, x, y, z) = x^{2} + y^{2} + z^{2} + w^{2} - xy + yz - yw - zw
    \label{ecuacion:1.5}
\end{equation}\\

tiene asociada la bigráfica\\

\begin{center}
\begin{tikzpicture}[scale=1.7]
    \node (v1) at (1.9914399682732804, 0.9921104441137079) {$x$};
    \node (v2) at (0.9905576274421105, 0.7146057713195639) {$y$};
    \node (v3) at (0.0, 0.9512105498862511) {$z$};
    \node (v4) at (0.26505088954607003, 0.0) {$w$};
    \draw (v1) -- (v2);
    \draw[dotted] (v2) -- (v3);
    \draw (v2) -- (v4);
    \draw (v3) -- (v4);
\end{tikzpicture}
\end{center}

y su matriz asociada\\

\begin{center}
\begin{equation*}
    A = \begin{bmatrix}
    2 & -1  &  0 &  0\\
   -1 &  2  &  1 & -1\\
    0 &  1  &  2 & -1 \\
    0 & -1  & -1 &  2
    \end{bmatrix}
\end{equation*}
\end{center}

$P = \begin{bmatrix}
   1 &  0  & 0 & 0\\
   0 &  1  & 0 & 0\\
   0 & -1  & 1 & 0 \\
   0 &  0  & 0 & 1
      \end{bmatrix}$ con inversa  $P^{-1} =  \begin{bmatrix}
   1 &  0  & 0 & 0\\
   0 &  1  & 0 & 0\\
   0 &  1  & 1 & 0 \\
   0 &  0  & 0 & 1
           \end{bmatrix}$
           
\begin{equation*}
\begin{split}
P^{T}~A~P & = \begin{bmatrix}
                1 &  0  &  0 & 0\\
                0 &  1  & -1 & 0\\
                0 &  0  &  1 & 0 \\
                0 &  0  &  0 & 1
             \end{bmatrix}
             \begin{bmatrix}
                2 & -1  &  0 &  0\\
               -1 &  2  &  1 & -1\\
                0 &  1  &  2 & -1 \\
                0 & -1  & -1 &  2
           \end{bmatrix}
           \begin{bmatrix}
                1 &  0  & 0 & 0\\
                0 &  1  & 0 & 0\\
                0 & -1  & 1 & 0 \\
                0 &  0  & 0 & 1
            \end{bmatrix}\\
 & = \begin{bmatrix}
                2 & -1  &  0 &   0\\
               -1 &  1  & -1 &   0\\
                0 &  1  &  2 &  -1 \\
                0 & -1  & -1 &   2
           \end{bmatrix}
           \begin{bmatrix}
                1 &  0  & 0 & 0\\
                0 &  1  & 0 & 0\\
                0 & -1  & 1 & 0 \\
                0 &  0  & 0 & 1
            \end{bmatrix}\\
 & = \begin{bmatrix}
                2 & -1  &   0   &   0\\
               -1 &  2  &  -1  &   0\\
                0 & -1  &   2  &  -1 \\
                0 &  0  &  -1  &   2
           \end{bmatrix}
\end{split}
\end{equation*}

Entonces podemos concluir que $q$ es $\mathbb{Z}$-equivalente a la forma unitaria

\begin{equation*}
q'\left(x, y, z, w\right) = x^{2} + y^{2} + z^{2} + w^{2} - xy - yz - zw
\end{equation*}

con gráfica asociada

\begin{center}
\begin{tikzpicture}[baseline=(v1.base)]
 \centering% El subgrafo está centrado
    \node (v1) at (0, 0) {x};
    \node (v2) at (1, 0) {y};
    \node (v3) at (2, 0) {z};
    \node (v4) at (3, 0) {w};
    \draw (v1) -- (v2);
    \draw (v2) -- (v3); 
    \draw (v3) -- (v4);
\end{tikzpicture}
\end{center}

Esta gráfica es isomorfa a $\mathbb{A}_{4}$, por lo tanto ese el tipo Dynkin es $q_{\mathbb{A}_{4}}$.
\end{example}

\paragraph{}
Para demostrar el teorema \ref{teorema:1.6} lo dividiremos en dos partes:

\begin{enumerate}
    \item Demostramos que las gráficas de Dynkin son las únicas gráficas conexas de aristas sólidas que definen formas unitarias que son definidas positivas (las aristas múltiples las consideraremos como aristas con peso, ver corolario \ref{corolario:1.9}) .
    \item Demostramos que siempre es posible hacer cambios de variable enteros de tal manera que la gráfica resultante no contenga aristas solidas.
\end{enumerate}

La última afirmación será demostrada en \ref{sec:2.1}

\begin{lemma}
Los diagramas de Dynkin son las únicas gráficas conexas de aristas sólidas que tienen asociadas formas unitarias definidas positivas.
\label{lema:1.7}
\end{lemma}

\paragraph{}
Para comenzar necesitamos convencernos en verdad de que los diagramas de Dynkin definen formas unitarias definidas positivas. Comencemos con las gráficas de tipo $\DynA_{n}$:

$$x_{1}\rule[1mm]{.1cm}{0.4pt} x_{2} \rule[1mm]{.1cm}{0.4pt}\cdots\rule[1mm]{.1cm}{0.4pt} x_{n}$$

\paragraph{}
Consideremos  la identidad:

\begin{equation*}
    \frac{1}{2}\left(x_{i} - x_{j}\right)^{2} = \frac{1}{2}x_{i}^{2} - x_{i}x_{j} + \frac{1}{2}x_{j}^{2}
\end{equation*}

\paragraph{}
Podemos reescribir la forma unitaria asociada a $\DynA_{n}$ como

\begin{eqnarray*}
 q_{\DynA_{n}}(\overrightarrow{x}) &  =  & \sum_{i=1}^{n}x_{i}^{2} - \sum_{i=1}^{n-1} x_{i}x_{i+1}\\
 &  =  & \frac{1}{2}x_{1}^{2} + \sum_{i=1}^{n-1}\left(\frac{1}{2}x_{i}^{2} + \frac{1}{2} x_{i+1}^{2}\right) + \frac{1}{2}x_{n}^{2} + \sum_{i=1}^{n-1} \left(-x_{i}x_{i+1}\right)\\
 &  =  & \frac{1}{2}x_{1}^{2} + \sum_{i=1}^{n-1}\left(\frac{1}{2}x_{i}^{2} - x_{i}x_{i+1} + \frac{1}{2}x_{i+1}^{2} \right) + \frac{1}{2}x_{n}^{2}\\
 &  =  & \frac{1}{2}x_{i}^{2} + \sum_{i=1}^{n-1}\frac{1}{2}\left(x_{i} - x_{i+1}\right)^{2} + \frac{1}{2}x_{n}^{2}
\end{eqnarray*}

\paragraph{}
Por el teorema \ref{teorema:1.4} se concluye que $q_{\DynA_{n}}$ es definida positiva. La demostración para la forma unitaria $q_{\DynD_{n}}$ es similar solo que en este caso se usa la identidad:

\begin{equation*}
    \frac{1}{2}\left[\left(x_{3} - x_{2} - x_{1}\right)^{2} + \left(x_{2} - x_{1}\right)^{2}\right] = x_{1}^{2} + x_{2}^{2} + \frac{1}{2}x_{3}^{2} - x_{1}x_{3} - x_{2}x_{3}
\end{equation*}

\paragraph{}
de tal forma que 
\begin{eqnarray*}
 q_{\DynD_{n}}(\overrightarrow{x}) &  =  & \sum_{i=1}^{n}x_{i}^{2} - x_{1}x_{3} -  \sum_{i=2}^{n-1} x_{i}x_{i+1}\\
 &  =  & \frac{1}{2}\left[\left(x_{3} - x_{2} - x_{1}\right)^{2} + \left(x_{2} - x_{1}\right)^{2} + \sum_{i=3}^{n-1}\left(x_{i} - x_{i+1}\right)^{2}  + x_{n}^{2}\right]
\end{eqnarray*}

\paragraph{}
Para las gráficas $\DynE_{6}$, $\DynE_{7}$ y $\DynE_{8}$ usaremos el siguiente razonamiento:\\
Supongamos que $A_{\Delta}$ es la matriz asociada a la gráfica $\Delta$ (con $\Delta = \DynE_{6},\DynE_{7},\DynE_{8}$); si existe una matriz $R_{\Delta}$ triangular superior, con entradas positivas en la diagonal principal tal que $A_{\Delta} = R_{\Delta}^{T}R_{\Delta}$ entonces por teorema \ref{teorema:1.5} se sigue que $\Delta$ define una forma unitaria definida positiva.\\
En efecto tenemos que:
\begin{align*}
 A_{\DynE_{6}} &  =  \begin{bmatrix}
 2 & 0 & 0 & -1 & 0 & 0\\
 0 & 2 & -1 & 0 & 0 & 0\\
 0 & -1 & 2& -1 & 0 & 0\\
 -1 & 0 & -1 & 2 & -1 & 0\\
 0 & 0 & 0 & -1 & 2 & -1\\
 0 & 0 & 0 & 0 & -1 & 2\\
 \end{bmatrix}\\
 R_{\DynE_{6}} &  =   \begin{bmatrix}
 \sqrt{2} & 0 & 0 & -\frac{1}{\sqrt{2}} & 0 & 0\\
 0 & sqrt(2) & - \frac{1}{\sqrt{2}}& 0 & 0 & 0\\
 0 & 0 & \frac{\sqrt{3}}{\sqrt{2}} & -\frac{\sqrt{2}}{\sqrt{3}} & 0 & 0\\
 0 & 0 & 0 & \frac{\sqrt{5}}{\sqrt{6}} & -\frac{\sqrt{6}}{\sqrt{5}} & 0\\
 0 & 0 & 0 & 0 & \frac{2}{\sqrt{5}} & -\frac{\sqrt{5}}{2}\\
 0 & 0 & 0 & 0 & 0 & \frac{\sqrt{3}}{2}\\
 \end{bmatrix}
 \end{align*}\\
 \begin{align*}
 A_{\DynE_{7}} &  =  \begin{bmatrix}
 2 & 0 & 0 & -1 & 0 & 0 & 0\\
 0 & 2 & -1 & 0 & 0 & 0 & 0\\
 0 & -1 & 2 & -1 & 0 & 0 & 0\\
 -1 & 0 & -1 & 2 & -1 & 0 & 0\\
 0 & 0 & 0 & -1 & 2 & -1 & 0\\
 0 & 0 0 & 0 & 0 & -1 & 2 & -1\\
 0 & 0 & 0 & 0 & 0 & -1 & 2\\
 \end{bmatrix}\\
 R_{\DynE_{7}} &  =  \begin{bmatrix}
 \sqrt{2} & 0 & 0 & -\frac{1}{\sqrt{2}} & 0 & 0 & 0 & 0\\
 0 & sqrt(2) & - \frac{1}{\sqrt{2}}& 0 & 0 & 0 & 0 & 0\\
 0 & 0 & \frac{\sqrt{3}}{\sqrt{2}} & -\frac{\sqrt{2}}{\sqrt{3}} & 0 & 0 & 0 & 0\\
 0 & 0 & 0 & \frac{\sqrt{5}}{\sqrt{6}} & -\frac{\sqrt{6}}{\sqrt{5}} & 0 & 0 & 0\\
 0 & 0 & 0 & 0 & \frac{2}{\sqrt{5}} & -\frac{\sqrt{5}}{2} & 0 & 0\\
 0 & 0 & 0 & 0 & 0 & \frac{\sqrt{3}}{2} & -\frac{2}{\sqrt{3}}\\
 0 & 0 & 0 & 0 & 0 & 0 & 0 & \frac{\sqrt{2}}{\sqrt{3}}\\
 \end{bmatrix}
 \end{align*}\\
 \begin{align*}
 A_{\DynE_{8}} &  = \begin{bmatrix}
 2 & 0 & 0 & -1 & 0 & 0 & 0 & 0\\
 0 & 2 & -1 & 0 & 0 & 0 & 0 & 0\\
 0 & -1 & 2 & -1 & 0 & 0 & 0 & 0\\
 -1 & 0 & -1 & 2 & -1 & 0 & 0 & 0\\
 0 & 0 & 0 & -1 & 2 & -1 & 0 & 0 \\
 0 & 0 & 0 & 0 & -1 & 2 & -1 & 0\\
 0 & 0 & 0 & 0 & 0 & -1 & 2 & -1\\
 0 & 0 & 0 & 0 & 0 & 0 & -1 & 2\\
 \end{bmatrix}
\end{align*}
\begin{align*}
 R_{\DynE_{8}} &  = \begin{bmatrix}
 \sqrt{2} & 0 & 0 & -\frac{1}{\sqrt{2}} & 0 & 0 & 0 & 0 & 0\\
 0 & sqrt(2) & - \frac{1}{\sqrt{2}}& 0 & 0 & 0 & 0 & 0 & 0\\
 0 & 0 & \frac{\sqrt{3}}{\sqrt{2}} & -\frac{\sqrt{2}}{\sqrt{3}} & 0 & 0 & 0 & 0 & 0\\
 0 & 0 & 0 & \frac{\sqrt{5}}{\sqrt{6}} & -\frac{\sqrt{6}}{\sqrt{5}} & 0 & 0 & 0 & 0\\
 0 & 0 & 0 & 0 & \frac{2}{\sqrt{5}} & -\frac{\sqrt{5}}{2} & 0 & 0 & 0\\
 0 & 0 & 0 & 0 & 0 & \frac{\sqrt{3}}{2} & -\frac{2}{\sqrt{3}} & 0 & 0\\
 0 & 0 & 0 & 0 & 0 & 0 & 0 & \frac{\sqrt{2}}{\sqrt{3}} & - \frac{\sqrt{3}}{\sqrt{2}}\\
 0 & 0 & 0 & 0 & 0 & 0 & 0 & \frac{1}{\sqrt{2}} & 0\\
 \end{bmatrix}
\end{align*}

\paragraph{}
Aun no hemos terminado la demostración del lema \ref{lema:1.7} . Falta demostrar que los diagramas de Dynkin son las únicas bigráficas de aristas sólidas que definen formas unitarias definidas positivas. Comenzaremos con el siguiente lema, el cual muestra que toda bigráfica que tenga aristas con peso mayor que uno no corresponde a ninguna forma unitaria definida positiva.

\begin{lemma}
Si la forma unitaria $q(\overrightarrow{x})$ es definida positiva entonces $q_{ij} \in \{-1,0,1\}$ para todo $1\leq i \le j \leq n$.
\label{lema:1.8}
\end{lemma}

\begin{proof}
Denotemos con $ \overrightarrow{e}_{k} = \left(d_{1},d_{2}, \ldots , d_{n}\right)$ al vector dado por $d_{k} = 1$ y $d_{i} = 0$ para toda $i\neq k$. Si $q_{ij} \geq 2$ entonces $q(\overrightarrow{e}_{i} + \overrightarrow{e}_{j}) = 2 - q_{ij} \leq 0$ pero $\overrightarrow{e}_{i} - \overrightarrow{e}_{j} \neq \overrightarrow{0}, ~ i \neq j$. Si $q_{ij} \leq -2$ entonces $q\left(\overrightarrow{e}_{i}+ \overrightarrow{e}_{j}\right) = 2 - q_{i,j} \geq 4$ pero $\overrightarrow{e}_{i} - e_{j} \neq \overrightarrow{0}$. De lo anterior, ya que $q$ es definida positiva, entonces $|q_{ij}| \le 2$ para todo $1 \leq i < j \leq n$.
\end{proof}

\paragraph{}
Cada $|q_{ij}|$ nos dice el peso de las aristas que hay entre los vértices $x_{i}$ y $x_{j}$ de la gráfica $\textbf{B}_{q}$; por lo tanto si $|q_{ij}| \leq 1$ tenemos que $\textbf{B}_{q}$ es una bigráfica simple. Es decir que el lema \ref{lema:1.8} se puede reescribir como sigue:

\begin{corollary}
Si $q$ es una forma unitaria positiva entonces necesariamente $\textbf{B}_{q}$ es una bigráfica simple.
\label{corolario:1.9}
\end{corollary}

\paragraph{}
Con base en este corolario diremos que una forma unitaria $q$ es \textbf{simple} si su gráfica asociada $\textbf{B}_{q}$ es una bigráfica simple.\\
Ahora que hemos descartado a las bigráficas de aristas con peso mayor que uno el resto de la demostración es como sigue:

\begin{enumerate}
    \item Demostraremos que toda gráfica que contenga a un diagrama Euclidiano (Figura \ref{figura:1.3}) no define a una forma unitaria definida positiva
    \item Demostraremos que toda bigráfica con sólo aristas solidas que no sea un diagrama de Dynkin necesariamente contiene una sub-bigráfica Euclidiana.
\end{enumerate}

\begin{figure}
    \centering
    \begin{tabular}{ll}
    \hline
    Notación \vline & Gráfica\\ 
    \hline
    $\widetilde{\DynA_{m}}$&
    \begin{tikzpicture}[baseline=(v1.base)]
  \node (v0) at (0.16, -0.76) {};
  \node (v1) at (-1.0, -0.79) {};
  \node (v2) at (-1.56, 0.23) {};
  \node (v3) at (-0.97, 1.24) {};
  \node (v4) at (0.19, 1.26) {};
  \node (v5) at (0.74, 0.24) {};
  \draw (v0) -- (v1);
  \draw (v1) -- (v2);
  \draw (v2) -- (v3);
  \draw (v3) -- (v4);
  \draw (v4) -- (v5);
  \draw[dotted] (v5) -- (v0);
    \end{tikzpicture}\\
    \newline
    $\widetilde{\DynD_{m}}$ &
    \begin{tikzpicture}[baseline=(v1.base)]
    \node (v1) at (0, 0) {};
    \node (v2) at (1, 0) {};
    \node (v3) at (2, 0) {};
    \node (v4) at (5, 0) {};
    \node (v5) at (1, 1) {};
    \node (v6) at (6, 0) {};
    \node (v7) at (5, 1) {};
    \node[draw = none] (dots) at (3.5, 0) {$\ldots$};
    \draw (v1) -- (v2) -- (v3) -- (3, 0); 
    \draw (v5) -- (v2); 
    \draw (4, 0) -- (v4); 
    \draw (v4) -- (v6); 
    \draw (v4) -- (v7);
    \end{tikzpicture}\\
    \newline
    $\widetilde{\DynE_{6}}$ &
    \begin{tikzpicture} [baseline=(v1.base)]
    \node (v1) at (0, 0) {};
    \node (v2) at (1, 0) {};
    \node (v3) at (2, 0) {};
    \node (v4) at (3, 0) {};
    \node (v5) at (4, 0) {};
    \node (v6) at (2, 1) {};
    \node (v7) at (2, 2) {};
    \draw (v1) -- (v2) -- (v3) -- (v4);
    \draw (v6) -- (v3);
    \draw (v5) -- (v4);
    \draw (v6) -- (v7);
    \end{tikzpicture}\\
    \newline
    $\widetilde{\DynE_{7}}$ &
    \begin{tikzpicture} [baseline=(v1.base)]
    \node (v1) at (0, 0) {};
    \node (v2) at (1, 0) {};
    \node (v3) at (2, 0) {};
    \node (v4) at (3, 0) {};
    \node (v5) at (4, 0) {};
    \node (v7) at (5, 0) {};
    \node (v6) at (2, 1) {};
    \node (v8) at (-1, 0) {};
    \draw (v1) -- (v2) -- (v3) -- (v4);
    \draw (v6) -- (v3);
    \draw (v4) -- (v5);
    \draw (v5) -- (v7);
    \draw (v8) -- (v3);
    \end{tikzpicture}\\
    \newline
    $\widetilde{\DynE_{8}}$&
    \begin{tikzpicture} [baseline=(v1.base)]
    \node (v1) at (0, 0) {};
    \node (v2) at (1, 0) {};
    \node (v3) at (2, 0) {};
    \node (v4) at (3, 0) {};
    \node (v5) at (4, 0) {};
    \node (v7) at (5, 0) {};
    \node (v8) at (6, 0) {};
    \node (v6) at (2, 1) {};
    \node (v9) at (7, 0) {};
    \draw (v1) -- (v2) -- (v3) -- (v4);
    \draw (v6) -- (v3);
    \draw (v4) -- (v5);
    \draw (v5) -- (v7);
    \draw (v7) -- (v8);
    \draw (v9) -- (v8);
    \end{tikzpicture}
    \end{tabular} 
    \caption{Diagramas Extendidos de Dynkin\citep{BarotMichaelJesusJose}. La cantidad de vértices que tiene cada gráfica es $n = m + 1$.}
    \label{figura:1.3}
\end{figure}

\paragraph{}
Hasta ahora hemos asociado vértices con variables(\begin{tikzpicture}\node (v1) at (-0.0, 0.56) {x}; \end{tikzpicture} con x). Ahora a cada variable $x_{i}$ le asignamos el vértice $i$, convenimos que le asignamos un orden a las variables y que el vértice $i$ representa el lugar de la variable $x_{i}$ en el orden dado.

\begin{center}
\begin{tikzpicture}[scale=1.7]
  \node (v0) at (-0.56, 0.0) {$x$};
  \node (v1) at (-0.0, 0.56) {$y$};
  \node (v2) at (0.56, 0.0) {$z$};
  \node (v3) at (-0.0, -0.56) {$w$};
  \draw[dotted] (v0) -- (v3);
  \draw (v0) -- (v1);
  \draw (v0) -- (v2);
  \draw[dotted] (v1) -- (v2);
  \draw (v1) -- (v3);
  \draw (v2) -- (v3);
\end{tikzpicture}
\end{center}

representa el orden $x_{1} = x$, $x_{2} = y$, $x_{3} = z$, $x_{4} = w$.

\paragraph{}
Mostraremos que si $\textbf{B}_{q}$ contiene una subgráfica Euclidiana entonces existe un vector $\overrightarrow{x} \neq 0$ tal que $q\left(\overrightarrow{x}\right) = 0$, mostrando así que $q(\overrightarrow{x})$ no es definida positiva. Un simple cálculo nos muestra que esta evaluación produce un vector $\overrightarrow{x} \neq \overrightarrow{0}$ tal que $q\left(\overrightarrow{x}\right) = 0$.\\

\begin{figure}[h]
    \begin{subfigure}[b]{0.5\textwidth}
        \begin{minipage}{7cm}
        \centering% El subgrafo está centrado
        \begin{tikzpicture}
        \node (v0) at (1.56, 0.14) {1};
        \node (v1) at (1.02, 1.11) {1};
        \node (v2) at (-0.07, 1.31) {1};
        \node (v3) at (-0.91, 0.58) {1};
        \node (v4) at (-0.89, -0.54) {1};
        \node (v5) at (0.01, -1.16) {1};
        \node (v6) at (1.07, -0.86) {1};
        \draw (v0) -- (v1);
        \draw (v1) -- (v2);
        \draw (v3) -- (v2);
        \draw (v3) -- (v4);
        \draw (v4) -- (v5);
        \draw (v5) -- (v6);
        \draw[dotted] (v6) -- (v0);
        \end{tikzpicture}
        \end{minipage}
        \caption{$\widetilde{\DynA}_{m}$}
     \end{subfigure}
     \begin{subfigure}[b]{0.5\textwidth}
        \begin{minipage}{7cm}
        \centering% El subgrafo está centrado
        \begin{tikzpicture}
        \node (v0) at (0.17, -0.93) {1};
        \node (v1) at (1.27, -0.93) {2};
        \node (v2) at (2.37, -0.93) {3};
        \node (v3) at (3.47, -0.93) {4};
        \node (v4) at (3.47, 0.73) {5};
        \node (v5) at (4.57, -0.93) {6};
        \node (v6) at (5.67, -0.93) {7};
        \node (v7) at (6.77, -0.93) {8};
        \draw (v0) -- (v1);
        \draw (v1) -- (v2);
        \draw (v2) -- (v3);
        \draw (v3) -- (v4);
        \draw (v3) -- (v5);
        \draw (v6) -- (v7);
        \draw (v5) -- (v6);
        \end{tikzpicture}
        \end{minipage}
        \caption{$\widetilde{\DynE}_{7}$}
     \end{subfigure}
     \begin{subfigure}[b]{0.5\textwidth}
        \begin{minipage}{7cm}
        \centering% El subgrafo está centrado
        \begin{tikzpicture}
        \node (v0) at (3.47,  1.93) {1};
        \node (v1) at (1.27, -0.93) {2};
        \node (v2) at (2.37, -0.93) {3};
        \node (v3) at (3.47, -0.93) {4};
        \node (v4) at (3.47, 0.43) {5};
        \node (v5) at (4.57, -0.93) {6};
        \node (v6) at (5.67, -0.93) {7};
        \draw (v0) -- (v4);
        \draw (v1) -- (v2);
        \draw (v2) -- (v3);
        \draw (v3) -- (v4);
        \draw (v3) -- (v5);
        \draw (v5) -- (v6);
        \end{tikzpicture}
        \end{minipage}
        \caption{$\widetilde{\DynE}_{6}$}
     \end{subfigure}
     \begin{subfigure}[b]{0.5\textwidth}
        \begin{minipage}{7cm}
        \centering% El subgrafo está centrado
        \begin{tikzpicture}
        \node (v0) at (7.87, -0.93) {1};
        \node (v1) at (1.27, -0.93) {2};
        \node (v2) at (2.37, -0.93) {3};
        \node (v3) at (3.47, -0.93) {4};
        \node (v4) at (3.47, 0.73) {5};
        \node (v5) at (4.57, -0.93) {6};
        \node (v6) at (5.67, -0.93) {7};
        \node (v7) at (6.77, -0.93) {8};
        \node (v8) at (8.97, -0.93) {9};
        \draw (v0) -- (v7);
        \draw (v1) -- (v2);
        \draw (v2) -- (v3);
        \draw (v3) -- (v4);
        \draw (v3) -- (v5);
        \draw (v6) -- (v7);
        \draw (v5) -- (v6);
        \draw (v0) -- (v8);
        \end{tikzpicture}
        \end{minipage}
        \caption{$\widetilde{\DynD}_{m}$}
     \end{subfigure}
     \begin{center}
      \begin{subfigure}[b]{0.8\textwidth}
        \centering% El subgrafo está centrado
        \begin{tikzpicture}
        \node (v0) at (1.27, 0.83) {1};
        \node (v1) at (1.27, -0.73) {2};
        \node (v2) at (2.37, -0.13) {3};
        \node (v3) at (3.47, -0.13) {4};
        \node (v4) at (6.27, -0.13) {5};
        \node (v5) at (7.37, -0.73) {6};
        \node (v6) at (7.37, 0.83) {7};
        \node[draw = none] (dots) at (4.87, -0.13) {$\ldots$};
        \draw (v0) -- (v2);
        \draw (v1) -- (v2);
        \draw (v2) -- (v3) -- (4.57, -0.13);
        \draw (5.17, -0.13) -- (v4) -- (v5);
        \draw (v4) -- (v6);
        \end{tikzpicture}
        \caption{$\widetilde{\DynE}_{8}$}
      \end{subfigure}
     \end{center}
     \caption{Formas hipercríticas no negativas asociadas a los Diagramas Extendidos de Dynkin.\citep{BarotMichaelJesusJose}}
    \label{figura:1.4}
\end{figure}

Solamente falta demostrar que toda gráfica que no sea de Dynkin necesariamente contiene a una subgráfica Euclidiana. Sea $G$ una gráfica conexa de $n$ vértices distinta de $\DynA_{n}$, $\DynB_{n}$, $\DynD_{n}$, $\DynE_{6}$, $\DynE_{7}$ y $\DynE_{8}$. Si $G$ no es un árbol entonces $G$ contiene un ciclo; es decir contiene una subgráfica de $\widetilde{\DynA}_{m}$ para algún $m < n$. Si $G$ es un árbol, y dado que $G \neq \DynA_{n}$, entonces existe al menos un vértice $v$ de grado 3 o más. Claramente $v$ pertenece a una subgráfica de $\DynD_{r}$ para algún $r \leq n$, pero habíamos supuesto que $G \neq \DynD_{n}$ por lo tanto hay tres casos:

\begin{enumerate}
    \item Si $v$ tiene grado estrictamente mayor a 3 entonces $G$ contiene a $\widetilde{D}_{4}$
    \item Si $G$ contiene otro vértice $w$ de grado 3 o más, entonces $G$ contiene a $\widetilde{D}_{m}$ para algún $m < n$
    \item Si todos los demás vértices tienen grado menor a 3 entonces $G$ debe contener a $\DynE_{6}$ como subgráfica.
\end{enumerate}

\paragraph{}
 Ahora ya que habíamos supuesto que $G \neq \DynE_{6}$, por tanto $n > 6$. En este caso $G$ debe contener a $\widetilde{\DynE}_{6}$ o $\DynE_{7}$. Si tenemos que $G \neq \DynE_{7}$ entonces $n > 8$., de donde obtenemos que $G$ contiene a $\widetilde{\DynE}_{7}$ o $\DynE_{8}$, pero si $G \neq \DynE_{8}$ entonces $n > 8$ y por lo tanto $G$ contiene a $\widetilde{\DynE}_{8}$.\\
Resumiendo, las gráficas de Dynkin definen formas unitarias positivas, y cualquier otra gráfica conexa y de aristas solidas que no sea un diagrama de Dynkin necesariamente contiene una gráfica Euclidiana que la vuelve no positiva: por lo tanto las gráficas de Dynkin son las únicas gráficas conexas de aristas sólidas que definen formas unitarias definidas positivas. Esto concluye la demostración del lema \ref{lema:1.7}.\\
En la sección \ref{sec:2.1} se dará termino a la demostración del teorema \ref{teorema:1.6}
