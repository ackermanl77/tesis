\documentclass[12pt,spanish,openany,letterpaper,pagesize]{scrbook}
\usepackage[utf8]{inputenc} %fuentes
\usepackage{lmodern} %fuentes
\usepackage{hyperref}
\usepackage[T1]{fontenc} %fuentes
\usepackage[spanish]{babel}
\usepackage{epsfig}
\usepackage{epic}
\usepackage{eepic}
\usepackage[square, numbers, comma, sort&compress]{natbib} % Use the natbib reference package - read up on this to edit the reference style; if you want text (e.g. Smith et al., 2012) for the in-text references (instead of numbers), remove 'numbers' 
\usepackage{amsmath}
\usepackage{amssymb}
\usepackage{amsthm}
\usepackage{booktabs}
%\usepackage{mathtools}
\usepackage{amsfonts}
\usepackage{dashrule}
\usepackage{amsfonts}
\usepackage{marginnote}
\usepackage{threeparttable}
\usepackage{fancyhdr}
\usepackage{amscd}
\usepackage{here}
\usepackage{mathrsfs}
\usepackage{lscape}
\usepackage{tabularx}
\usepackage{subcaption}
\usepackage{longtable}
\usepackage{scrhack}
\usepackage{epsfig}
\usepackage{longtable}
\usepackage{tikz}
\usepackage{tkz-graph}
\usetikzlibrary{graphdrawing,positioning,graphs}
\usegdlibrary{layered, trees}
\usepackage{anyfontsize}
 %idoma, cargué los dos porque constantemente estoy cambiando
\usepackage{listings}
\usepackage{textgreek}
\usepackage{geometry}
\usepackage{multirow, array} % para las tablas
\usepackage{float} % para usar [H] 'numbers' 
\usepackage{stackengine}
\usepackage{multicol}%
\usetikzlibrary{arrows}
\usepackage{graphicx}
\usepackage{xcolor}
\usepackage{lmodern}
\selectlanguage{spanish}
\usepackage[linesnumbered,ruled, noline, noend, onelanguage]{algorithm2e}
%\floatname{algorithm}{Algoritmo}
 % mi archivo de traducción
\usepackage{rotating} %Para rotar texto, objetos y tablas seite. No se ve en DVI solo en PS. Seite 328 Hundebuch

\usepackage[
    type={CC},
    modifier={by-sa},
    version={4.0},
]{doclicense}

%\geometry{
%paperwidth=15.3cm,
%paperheight=23.2cm,
%top=1.5cm,
%bottom=.8cm,
%left=1.9cm,
%right=1.9cm
%}
\newcommand{\HRule}{\rule{\linewidth}{0.5mm}} % 

\DefineNamedColor{named}{Maroon} {cmyk}{0,0.87,0.68,0.32}

\SetKwIF{If}{ElseIf}{Else}{si}{}{si no}{en otro caso}{endif}
\SetKwFor{For}{para}{}{endfor}
\SetKwFor{While}{mientras}{}{endw}

\newcommand{\nosemic}{\renewcommand{\@endalgocfline}{\relax}}% Drop semi-colon ;
\newcommand{\dosemic}{\renewcommand{\@endalgocfline}{\algocf@endline}}% Reinstate semi-colon ;
\newcommand{\pushline}{\Indp}% Indent
\newcommand{\popline}{\Indm\dosemic}% Undent
\let\oldnl\nl% Store \nl in \oldnl
\newcommand{\nonl}{\renewcommand{\nl}{\let\nl\oldnl}}% Remove line number for one line

                        %se usa junto con \rotate, \sidewidestable ....
% Como numerar las ecuaciones, figuras y tablas

\renewcommand{\theequation}{\thechapter-arabic{equation}}
\renewcommand{\thefigure}{\textbf{\thechapter-\arabic{figure}}}
\renewcommand{\thetable}{\textbf{\thechapter-\arabic{table}}}

%Estilo de los encabezados

%Options: Sonny, Lenny, Glenn, Conny, Rejne, Bjarne, Bjornstrup
\usepackage[Bjornstrup]{fncychap}

\KOMAoptions{DIV=13}%Margenes 4-15 (menos numero es menos margen)
%\usepackage{showframe}


\KOMAoptions{headlines=2.1}
\KOMAoptions{footsepline=true}

 
\newcommand{\chapternumbering}[1]{% 
  \setcounter{chapter}{0}% 
   \renewcommand{\thechapter}{\csname #1\endcsname{chapter}}}

\addtolength{\headwidth}{0cm}
%\unitlength1mm %Define la unidad LE para Figuras
\marginparwidth0cm
\parindent0cm %Define la distancia de la primera linea de un parrafo a la margen

%Para tablas,  redefine el backschlash en tablas donde se define la posici\'{o}n del texto en las
%casillas (con \centering \raggedright o \raggedleft)
\newcommand{\PreserveBackslash}[1]{\let\temp=\\#1\let\\=\temp}
\let\PBS=\PreserveBackslash

%Espacio entre lineas
\setlength{\parskip}{.8em}
\renewcommand{\baselinestretch}{.8}

%\renewcommand{\thesection}{\arabic{section}}

\numberwithin{equation}{chapter}
\numberwithin{algocf}{chapter}
%\numberwithin{algorithm}{chapter}


%%% Coloring the comment as blue
\newcommand\mycommfont[1]{\footnotesize\ttfamily\textcolor{blue}{#1}}
\SetCommentSty{mycommfont}


\theoremstyle{plain}
\newtheorem{definition}{Definición}[chapter]
\newtheorem{theorem}{\underline{\textbf{Teorema}}}[chapter]
\newtheorem{lemma}[theorem]{\underline{\textbf{Lema}}}
\newtheorem{corollary}[theorem]{\underline{\textbf{Corolario}}}
\counterwithin{figure}{chapter}
\counterwithin{table}{chapter}
\newtheorem{example}{Ejemplo}[chapter]

\theoremstyle{definition}
\newtheorem{defn}[equation]{Definition}
\newtheorem{prob}[equation]{Problem}


\theoremstyle{remark}
\newtheorem{note}[equation]{Note}
\newtheorem{rem}[equation]{Remark}


%Neuer Befehl f\"{u}r die Tabelle Eigenschaften der Aktivkohlen
\newcommand{\arr}[1]{\raisebox{1.5ex}[0cm][0cm]{#1}}

%Neue Kommandos
\usepackage{Befehle}
\usepackage{pythonhighlight}

%Trennungsliste
\hyphenation {Reaktor-ab-me-ssun-gen Gas-zu-sa-mmen-set-zung
Raum-gesch-win-dig-keit Durch-fluss Stick-stoff-gemisch
Ad-sorp-tions-tem-pe-ra-tur Klein-schmidt
Kohlen-stoff-Mole-kular-siebe Py-rolysat-aus-beu-te
Trans-port-vor-gan-ge}

%\hypersetup{colorlinks,linkcolor={blue},citecolor={blue},urlcolor={red}}
\hypersetup{urlcolor=black, colorlinks=true} % Colors hyperlinks in blue - change to black if annoying