\begin{appendix}
\chapter{Anexo: Nombrar el anexo A de acuerdo con su contenido}\label{AnexoA}
Los Anexos son documentos o elementos que complementan el cuerpo de la tesis o trabajo de investigaci\'{o}n y que se relacionan, directa o indirectamente, con la investigaci\'{o}n, tales como acetatos, cd, normas, etc.\\

\chapter{Anexo: Nombrar el anexo B de acuerdo con su contenido}
A final del documento es opcional incluir \'{\i}ndices o glosarios. \'{E}stos son listas detalladas y especializadas de los t\'{e}rminos, nombres, autores, temas, etc., que aparecen en el mismo. Sirven para facilitar su localizaci\'{o}n en el texto. Los \'{\i}ndices pueden ser alfab\'{e}ticos, cronol\'{o}gicos, num\'{e}ricos, anal\'{\i}ticos, entre otros. Luego de cada palabra, t\'{e}rmino, etc., se pone coma y el n\'{u}mero de la p\'{a}gina donde aparece esta informaci\'{o}n.\\

\chapter{Anexo: Nombrar el anexo C de acuerdo con su contenido}
MANEJO DE LA BIBLIOGRAF\'{I}A: la bibliograf\'{\i}a es la relaci\'{o}n de las fuentes documentales consultadas por el investigador para sustentar sus trabajos. Su inclusi\'{o}n es obligatoria en todo trabajo de investigaci\'{o}n. Cada referencia bibliogr\'{a}fica se inicia contra el margen izquierdo.\\

La NTC 5613 establece los requisitos para la presentaci\'{o}n de referencias bibliogr\'{a}ficas citas y notas de pie de p\'{a}gina. Sin embargo, se tiene la libertad de usar cualquier norma bibliogr\'{a}fica de acuerdo con lo acostumbrado por cada disciplina del conocimiento. En esta medida es necesario que la norma seleccionada se aplique con rigurosidad.\\

Es necesario tener en cuenta que la norma ISO 690:1987 (en Espa\~{n}a, UNE 50-104-94) es el marco internacional que da las pautas m\'{\i}nimas para las citas bibliogr\'{a}ficas de documentos impresos y publicados. A continuaci\'{o}n se lista algunas instituciones que brindan par\'{a}metros para el manejo de las referencias bibliogr\'{a}ficas:\\

\begin{center}
\centering%
\begin{tabular}{|p {7.5 cm}|p {7.5 cm}|}\hline
\arr{Instituci\'{o}n}&Disciplina de aplicaci\'{o}n\\\hline%
Modern Language Association (MLA)&Literatura, artes y humanidades\\\hline%
American Psychological Association (APA)&Ambito de la salud (psicolog\'{\i}a, medicina) y en general en todas las ciencias sociales\\\hline
Universidad de Chicago/Turabian &Periodismo, historia y humanidades.\\\hline
AMA (Asociaci\'{o}n M\'{e}dica de los Estados Unidos)&Ambito de la salud (psicolog\'{\i}a, medicina)\\\hline
Vancouver &Todas las disciplinas\\\hline
Council of Science Editors (CSE)&En la actualidad abarca diversas ciencias\\\hline
National Library of Medicine (NLM) (Biblioteca Nacional de Medicina)&En el \'{a}mbito m\'{e}dico y, por extensi\'{o}n, en ciencias.\\\hline
Harvard System of Referencing Guide &Todas las disciplinas\\\hline
JabRef y KBibTeX &Todas las disciplinas\\\hline
\end{tabular}
\end{center}

Para incluir las referencias dentro del texto y realizar lista de la bibliograf\'{\i}a en la respectiva secci\'{o}n, puede utilizar las herramientas que Latex suministra o, revisar el instructivo desarrollado por el Sistema de Bibliotecas de la Universidad Nacional de Colombia\footnote{Ver: www.sinab.unal.edu.co}, disponible en la secci\'{o}n "Servicios", opci\'{o}n "Tr\'{a}mites" y enlace "Entrega de tesis".

\end{appendix}