\subsection{La idea general de el algoritmo de componentes triconexas}

\begin{algorithm}
\begin{algorithmic}[1]
\WHILE {$K \neq \{ \langle v \rangle \}$}
\STATE Elegir un símplice $\sigma_i$ de $K$ que sea maximal y que contenga una cara libre $\delta\sigma_i$.
\IF{no hay ningún símplice de esas características}
\RETURN \FALSE
\ELSE
\STATE $K \leftarrow K \setminus \{\sigma_i, \delta\sigma_i\}$
\ENDIF
\ENDWHILE
\RETURN \TRUE
\end{algorithmic}
\caption{Contracción de caras libres maximales}\label{alg:algoritmoRaro}
\end{algorithm}






