\documentclass[11pt]{beamer}
\usetheme{PaloAlto}
%NO NECESITAS HACER OTRO ARCHIVO PARA CAMBIAR DE TEMA, SI QUIERES CAMBIAR
%SOLO BORRA EEL DE ARRIBA Y ESCRRIBE EL QUE TE GUSTE, OBVIAMENTE DE LOS QUE TIENE MORRO
\usepackage[utf8]{inputenc}
\usepackage[spanish]{babel}
\usepackage{amsmath}
\usepackage{amsfonts}
\usepackage{amssymb}
%NECESITAS QUITAR LOS "%" DE ABAJO PARA HABILITAR LAS FUNCIONES QUE QUIERAS MORRO
\author{MATECHOLO}
\title{UNA NO MUY LARGA INTRODUCCIÓN A BEAMER}
%\setbeamercovered{transparent} 
%\setbeamertemplate{navigation symbols}{} 
%\logo{} 
%\institute{} 
%\date{} 
%\subject{} 
\begin{document}

%ESTE OBVIO ES PARA LA PORTADA DEL BEAMER
\begin{frame}
\titlepage
\end{frame}

%Y ESTE LA TABLA DE CONTENIDOS
\begin{frame}
\tableofcontents
\end{frame}

%ESTO DE "SECCION" ES LO QUE APARECE EN LAS LA TABLA DE CONTENIDO
\section{la primer sección}
% Y ESTA OBVIAMENTE UNA SUBSECCIÓN
\subsection{la primer subsection}
%CADA FRAME ES UNA DIAPOSITIVA
\begin{frame}{AQUI VAN LOS TITULOS DE LAS DIAPOOSTIVAS}
Y AQUI UN CONTENIDO NORMAL.\\
ES COMO PARA INTRODUCCIONES O TEXTOS COMUNES.
\end{frame}

\begin{frame}
ESTA ES UNA DIAPOSITIVA SIN TITULO.\\
VES?.
\end{frame}

\begin{frame}{DIAPOSITIVAS RÁPIDAS}
PARA HACER MÁS RÁPIDO UNA DIAPOSITIVA SOLO ESRIBE LA BARRA INVERTIDA Y  begin LE DAS ENTER A LAS OPCION QUE TE RECOMIENDA Y DESPUÉS f Y LE DAS ENTER, ES COMO EL PREDICTOR DE PALABRAS DE GOOGLE.\\
SI NO LO VES YA TE DARÁS CUENTA ESE.
\end{frame}

\section{FÓRMULAS}
\begin{frame}
PARA UNA FÓRMULA MATEMÁTICA USA ESTO \[AQUI\:VA\:LA\:FORMULA\] ASI LA FORMULA APARECE EN UNA LÍNEA APARTE Y CENTRADA Y NO NECESITAS PONER EL SIGNO DE PESOS PARA PONER UN CARACTER ESPECIAL, COMO LETRA GRIEGAS O ASÍ. \\
LA BARRA INVERTIDA Y LOS DOS PUNTOS ES PARA DAR UN ESPACIO YA QUE AHÍ NO LES DA ESPACIO AUNQUE LOS SEPARES UN BUEN.\quad ESTO ES PARA UN ESPACIO MÁS GRANDE Y \qquad ES PARA UNO AÚN MÁS GRANDE.\\
RECUERDA DEJAR EL ESPACIO ENTRE EL quad Y EL TEXTO O TE PRROVOCARA UN ERROR.\\
\begin{center}
IGUAL PARA LOS OTROS
\end{center}
\end{frame}

\section{ESPACIOS Y OTRAS COSAS}
\begin{frame}[shrink]
LA ACCIÓN DE  shrink ES PARA CUANDO PONES DEMASIADO TEXTO EN UNA SOLA DIAPOSITIVA.\

ESTE LO AJUSTA, PERO SOLO A LO LARGO Y NO A LO ANCHO. \

PERO SI PONES POCO TEXTO EN UNOS TEMAS SE VE UN POCO RARO Y NO MUY ESTETICO.\\
\vspace{3mm}
LA ANTERIOR ACCIÓN ES DAR UN ESPACIO VERTICAL.\\
\vspace{4mm}
Y TU LE DICES QUE TAN GRANDE.
\begin{center}
CON ESTO ESCRIBES EN EL CENTRO
\end{center}
\begin{flushright}
Y CON ESTE A LA IZQUIERDA.
\end{flushright}
\end{frame}

\subsection{LAS OTRAS COSAS}
\begin{frame}{PARA TEOREMAS Y ESAS COSAS}
ASI COMIENZA UN TEOREMA
\begin{theorem}
Y ESCRIBES EL TEOREMA
\end{theorem}
Y ASÍ LAS DEMO-S-TRACIONES
\begin{proof}
LA DEMOSTRACIÓN ES TRIVIAL :P
\end{proof}
\end{frame}

\begin{frame}
PERO SI SON MUY LARGOS LOS PUEDES ESCRIBIR EN BLOQUES
\begin{block}{POR SI LE QUIERES PONER TITULOS}
Y COMIENZAS A ESCRIBIR
\end{block}
\begin{block}{DEMOSTRACIÓN}
IGUAL PARA LAS GRANDE DEMOSTRACIONES NO USE LA OPCIÓN DE proof USÉ LOS DE BLOCK Y PARA EL CUADRITO ALGO ASI
\begin{flushright}
$\square$
\end{flushright}
\end{block}
SOLO QUE CAMBIA EL COLOR DEL CUADRO EN ALGUNOS TEMAS, PERO CASI NADIE SE DA CUENTA.
\end{frame}

\subsection{LOS BLOCKS}
\begin{frame}
\begin{block}

SI NO LE PONES TITULO AL BLOCK, ENTONCES DEBES DEJAR UN ESPACO O SOLO LA PRIMER LETRA LA PONDRA COMO TÍTULO
\end{block}
\begin{block}
ALGO ASÍ
\end{block}
\end{frame}

\section{UN EJEMPLO DE LO QUE USE}
\begin{frame}{Operador integral singular}
Tomaremos en cuenta los operadores integrales singulares de la forma \[(\textit{S}_{\Gamma}\varphi)(\textit{t})=\frac{1}{\pi \imath} \int_{\Gamma} \frac{\varphi(\tau)}{\tau-\textit{t}}d\tau\]
donde $\textit{S}_{\Gamma}$ es entendida en el sentido del valor principal de Cauchy.\\
Sea $\Gamma$ una curva orientada en contra de las manecillas del reloj. \\
\vspace{3mm}
Llamaremos a $\Gamma$ una \emph{curva cerrada} si separa la extensión del plano complejo $\textbf{C}^\infty=\textbf{C}\cup{\infty}$ en dos dominios $\textit{F}_{\Gamma}^{+}$ y $\textit{F}_{\Gamma}^{-}$ tal que $\Gamma$ es frontera de ambos dominios. Asumiremos que $z=0\in F_{\Gamma}^{+}$ y $z=\infty\in F_{\Gamma}^{-}$.\\
\vspace{3mm}
 Denotaremos a $\textit{R}(\Gamma)$ como el conjunto de todas funciones racionales que no tienen polos en la curva $\Gamma$, mientras que $\textit{R}_{\pm}(\Gamma)$ será el conjunto de todas funciones racionales de los cuales sus polos son localizados en $\textit{F}_{\Gamma}^{\pm}$.
\end{frame}

\begin{frame}
 \begin{theorem}[2]
	Sea $A=r_{1}P_{\Gamma}+r_{2}Q_{\Gamma}$ un operador con coeficientes $r_{1},r_{2}\in R(\Gamma)$. Es 			necesario y suficiente para que el operador $A$ sea invertible de un lado en el espacio $L_{p}(\Gamma,		\rho)$ son las validaciones de las condiciones \[r_{j}(t)\neq 0\quad(j=1,2;t\in \Gamma).\] Si se 	satisfacen estas condiciones, entonces el operador $A$ es invertible, solo es invertible por la izquierda 	o solo es invertible por la derecha dependiendo de si el número \[k=ind\:(r_{1}/r_{2})\] es igual a 	cero, positivo o negativo.
 \end{theorem}
\end{frame}

\begin{frame}[shrink]
\begin{block}{Demostración}
$\Leftarrow$)\

Supongamos que $r_{j}\neq 0$ ($j=1,2;t\in\Gamma$) y $r=r_{-}t^{k}r_{+}$ es la factorización de la función $r=r_{1}/r_{2}$. Entonces el operador $A$ tiene la forma \[A=r_{2}(r_{-}t^{k}r_{+}P_\Gamma+Q_{\Gamma})=r_{2}r_{-}(t^{k}r_{+}P_{\Gamma}+r_{-}^{-1}Q_{\Gamma})\] y lo podemos reescribir en la forma \[A=r_{2}r_{-}(t^{k}P_{\Gamma}+Q_{\Gamma})(r_{+}P_{\Gamma}+r_{-}^{-1}Q_{\Gamma}).\] Los operadores $r_{2}r_{-}I$ y $r_{+}P_{\Gamma}+r_{-}^{-1}Q_{\Gamma}$ son invertibles, donde \[(r_{2}r_{-}I)^{-1}=r_{2}^{-1}r_{-}^{-1}I,\quad(r_{+}P_{\Gamma}+r_{-}^{-1}Q_{\Gamma})^{-1}=r_{+}^{-1}P_{\Gamma}+r_{-}Q_{\Gamma}.\] El operador $t^{k}P_{\Gamma}+Q_{\Gamma}$ es invertible solo por la izquierda para $k>0$ y es solo invertible por la derecha para $k<0$. La correspondiente inversa a el operador es $t^{-k}P_{\Gamma}+Q_{\Gamma}$.
\end{block}
\end{frame}

\begin{frame}[shrink]
\begin{block}

$\Rightarrow$)\

Primero consideremos el caso $r_{2}=1$. Supongamos que $r_{1}$ tiene un cero en algún punto $t_{0}\in\Gamma$. Lo expresamos en la forma $r_{1}=(t-t_{0})s$ y $r_{1}=(t^{-1}-t_{0}^{-1})q$. Para el operador $A=r_{1}P_{\Gamma}+Q_{\Gamma}$, tenemos las ecuaciones
\[A=(sP_{\Gamma}+Q_{\Gamma})((t-t_{0})P_{\Gamma}+Q_{\Gamma})\] y \[A=((t^{-1}-t_{0}^{-1})P_{\Gamma}+Q_{\Gamma})(P_{\Gamma}qP_{\Gamma}+(t^{-1}-t_{0}^{-1})Q_{\Gamma}qP_{\Gamma}+Q_{\Gamma}).\] El operador $A$ por la observación 1 no puede ser invertible pues los operadores $(t-t_{0})P_{\Gamma}+Q_{\Gamma}$ y $(t^{-1}-t_{0}^{-1})P_{\Gamma}+Q_{\Gamma}$ no son invertibles.\

De manera similar si el operador tiene la forma $P_{\Gamma}+r_{2}Q_{\Gamma}$ y es invertible de un lado, entonces $r_{2}$ no desaparece en $\Gamma$.
\end{block}
\end{frame}

\begin{frame}[shrink]
\begin{block}

En el caso general, $A=r_{1}P_{\Gamma}+r_{2}Q_{\Gamma}$. Escogemos las funciones $\chi_{1},\chi_{2}\in R(\Gamma)$ que no desaparecen en $\Gamma$. Más aún,
\[\chi_{1}r_{1}\in R_{+}(\Gamma),\quad \chi_{2}r_{2}\in R_{-}(\Gamma).\] Bajo estas condiciones el operador $A$ puede ser representado en dos formas \[A=\chi_{1}^{-1}(P_{\Gamma}+\chi_{1}r_{2}Q_{\Gamma})(\chi_{1}r_{1}P_{\Gamma}+Q_{\Gamma})\] y \[A=\chi_{2}^{-1}(\chi_{2}r_{1}P_{\Gamma}+Q_{\Gamma})(P_{\Gamma}+\chi_{2}r_{2}Q_{\Gamma}).\] Por lo que si $A$ es invertible por la izquierda, entonces los operadores $\chi_{2}r_{1}P_{\Gamma}+Q_{\Gamma}$ y $P_{\Gamma}+\chi_{1}r_{2}Q_{\Gamma}$ son invertibles por la izquierda. Similarmente si es invertible por la derecha, entonces los operadores $P_{\Gamma}+\chi_{2}r_{2}Q_{\Gamma}$ y $\chi_{1}r_{1}P_{\Gamma}+Q_{\Gamma}$ también son invertibles por la derecha.\

Por lo anterior probado, podemos deducir que las funciones $r_{1}$ y $r_{2}$ no desaparecen en $\Gamma$.\qquad\qquad\qquad\qquad\qquad\qquad\qquad\qquad$\square$
\end{block}
\end{frame}

\end{document}